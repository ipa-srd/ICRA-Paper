\begin{abstract}
%\boldmath

The ability to collaborate with humans is a key requirement of mobile robots to be versatile in a wide range of applications.
From a technological perspective, this requires the robot to be aware of people in its close to midrange workspace.
Due to safety purposes, most of todays mobile robots are equipped with 2D safety laser scanners mounted at leg-height.
Using this sensor data for perceiving the environment without needing additional sensors is highly beneficial in terms of economic reasons and flexibility of the hardware design. \\
This study presents a new approach for people detection and motion prediction in the robots workspace by solely processing 2D laser scan data.
Making use of the latest advances in neural networks, fast scalable applications can be modeled without having knowledge about the inner workings. 
First, we present a leg detection tool, that operates on the raw laser scan data, classifying each point as human or non-human.
The output is further used to learn localizing humans and predicting their trajectories.
In our experiments, we show how our approach outperforms a state-of-the-art leg detection approach and is able to precisely localize humans as well as predicting their motion for a short prediction horizon.


\end{abstract}



% IEEEtran.cls defaults to using nonbold math in the Abstract.
% This preserves the distinction between vectors and scalars. However,
% if the journal you are submitting to favors bold math in the abstract,
% then you can use LaTeX's standard command \boldmath at the very start
% of the abstract to achieve this. Many IEEE journals frown on math
% in the abstract anyway.

% Note that keywords are not normally used for peerreview papers.
\begin{IEEEkeywords}
Human Detection and Tracking,
Deep Learning in Robotics and Automation,
Recognition
\end{IEEEkeywords}






% For peer review papers, you can put extra information on the cover
% page as needed:
% \ifCLASSOPTIONpeerreview
% \begin{center} \bfseries EDICS Category: 3-BBND \end{center}
% \fi
%
% For peerreview papers, this IEEEtran command inserts a page break and
% creates the second title. It will be ignored for other modes.
\IEEEpeerreviewmaketitle
