\section{Introduction}

Autonomous robots are making their way from the production industry to the service industry. This change introduces a new set of problems, that deal with interactions of people and robots. Some of these problems can be simplified by first detecting persons, which then allows for example, to follow people, approach or evade them while navigating through an environment.

All of these problems can be solved with model-based solutions, but since neural networks are becoming more and more viable, we introduce a new approach, that does not require knowledge about people or their movements. Instead, the network will find characteristics on its own and place unique identifiers on persons. This is achieved by clustering the laser scan, but using additional information we find from the trained network.

Having the information and history of positions of people, we can learn trajectories, behaviour and intentions when the robot is serving customers. We provide a way to model the trajectories using Long short-term memory (LSTM) cells in combination with a Mixture density layer, which outputs a set of normal distributions, similar to \textit{paper from graves}.

We will go through the models that are used to train the program, as well as some benchmarking and comparisons with similar projects.