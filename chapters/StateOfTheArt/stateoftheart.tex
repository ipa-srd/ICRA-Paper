\section{State of the art}
\label{sec:sota}

Evaluating Lidar data for navigation tasks in mobile robotics is widespread due to facts mentioned in Sec. \ref{sec:intro} but also due to the simple interpretation of the data and the low computational power required.
In \cite{1013691}, lidar data is used for people dection and tracking where laser range measurements are grouped into blobs and objects. 
Moving objects are considered as people with the downside of not detecting non-moving persons.
In order to improve on this idea, a combination of Lidar and vision data \cite{kleinehagenbrock2002person} can be applied. 
Information of leg positions are here extracted from the lidar sensor, while a camera provides additional information through skin-colored face detection. 
Arras et al. proposed a machine learning algorithm \cite{Arras07usingboosted}, where pre-defined geometrical features are trained in a supervised learning approach. 
It was further adapted in \cite{weinrich2014people} to expand the detection in retirement homes, allowing the classification of people with different walking aids. 
The algorithm allows to add additional features, which are weighted and therefore filtered by importance.
However, the expert knowledge of geometrical features pose a downside to the algorithm as well as the computational power required during evaluation.\\
Convolutional neural networks present a way to learn features without prior expertise. 
Since their introduction \cite{lecun_gradient-based_1998}, they are the go-to standard for object detection \cite{krizhevsky_imagenet_2012}, due to their versatility and their simple and fast implementation. 
Convolutional networks can be further expanded for segmentation of images \cite{long2015fully}, which gives a way to keep the input dimensionality while simultaneously detect, localize and label objects.
In the presented approach, convolutional networks are used to detect legs from raw scan data.\\
Predicting human motion is a difficult task due to the complex behavior of humans and the variety of possible manoeuvres.
For short term motion prediction of objects like vehicles, a good estimates is often achieved by applying the current motion state to the vehicles motion model.
For humans however, the current motion state is not that conclusive on the future motion and an appropriate model is hard to design due to the complex kinematics. 
Due to those constraints, we also aim for a learning based approach for the task of motion prediction.
Machine learning based trajectory prediction has already been achieved in \cite{graves2013generating}, where handwriting is generated using a mixture density network (MDN) \cite{bishop1994mixture}. 
These networks provide a way to model the most probable outcome as a function of multiple density terms. 
In order to predict from a history of input values, LSTM is included in \cite{hochreiter1997long}. 
This allows a way to store previous inputs and therefore predict outcomes when similarities are found. 
People based trajectory prediction has previously been investigated in \cite{alahi2016social} using a similar approach.
We build open these approaches, by applying LSTM and MDN on trajectory prediction using unsupervised learning, requiring only positions of people as an input.
