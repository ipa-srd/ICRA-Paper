\section{State of the art}

LIDAR scanners pose an attractive way to scan the environment, due to the simple interpretation of the data and the low computational power required. 
The idea to extract information about people has been introduced in \cite{1013691} where laser range measurements \abbrev{LRM} are grouped into blobs and objects. 
Moving objects could then be considered as people while any other object was not of interest. 
In order to detect still-standing people, different approaches \cite{kleinehagenbrock2002person}, \cite{aguirre2014leg} used a combination of \abbrev{LRM} and imaging sensors.

Arras et al. proposed a machine learning approach \cite{Arras07usingboosted}, where pre-defined geometrical features were trained using supervised learning. 
It was further adapted in \cite{weinrich2014people} to expand the detection in retirement homes, allowing the classification of people with different walking aids. 
This approach is very versatile, as it can easily be adapted by simply adding further features for a given class.

We improve on this idea, by using neural networks as the machine learning technique.
This way, no knowledge about shapes or behaviour is required, which makes this approach easy to implement and also requires less computational power. 
For the human eye, one big factor in the detection of legs on a laser scan is the movement of people. 
By providing the network with multiple frames at once, we can therefore increase the accuracy, as the system will then take motion into account.\\

todo: expand section with some work of vision based object detection, convolutional networks and other related work that was of interest for the selected approach