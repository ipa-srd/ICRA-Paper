\section{State of the art}

LIDAR scanners pose an attractive way to scan the environment, due to the simple interpretation of the data and the low computational power required. The idea to extract information about people has been introduced in \cite{1013691} where laser range measurements \abbrev{LRM} are grouped into blobs and objects. This way, objects that were moving were considered people. Other approaches were also able to detect still-standing people by including camera sensors \cite{1013691} \cite{aguirre2014leg} into their computations.

Arras et al. proposed a machine learning approach, where pre-defined geometrical features were trained using supervised learning. This approach has been found to be versatile and was also adapted in \cite{weinrich2014people} to expand the detection in retirement homes, allowing the classification of people with different walking aids. Including movement of people, however, increased the detection error, although it is an important factor.

We improve on this idea, by using neural networks as the machine learning technique. This way, no knowledge about shapes or behaviour is required, which makes this approach easy to implement and also requires less computational power. By providing the network with multiple frames at once, we can increase the accuracy, as the system will learn to detect movement.