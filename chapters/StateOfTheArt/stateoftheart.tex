\section{State of the art}

LIDAR scanners pose an attractive way to scan the environment, due to the simple interpretation of the data and the low computational power required.
The idea to extract people from this input has been introduced in \cite{1013691} where laser range measurements are grouped into blobs and objects. 
Moving objects are considered as people with the downside of not detecting immovable objects.
In order to improve on this idea, a combination of LIDAR and vision data \cite{kleinehagenbrock2002person} can be applied. Information of leg positions are here extracted from the laser scanner, while a camera provides additional information through skin-colored face detection. 
Arras et al. proposed a machine learning algorithm \cite{Arras07usingboosted}, where pre-defined geometrical features are trained in a supervised learning approach. 
It was further adapted in \cite{weinrich2014people} to expand the detection in retirement homes, allowing the classification of people with different walking aids. 
The algorithm allows to add additional features, which are weighted and therefore filtered by importance.
However, the expert knowledge of geometrical features pose a downside to the algorithm as well as the computational power required during evaluation. Convolutional neural networks therefore present a way to learn features without prior expertise. Since their introduction \cite{lecun_gradient-based_1998}, they are the go-to standard for object detection \cite{krizhevsky_imagenet_2012}, due to their versatility and their simple and fast implementation. Convolutional networks can be further expanded for segmentation of images \cite{long2015fully}, which gives a way to keep the input dimensionality while simultaneously detect and label objects.

Trajectory prediction has already been achieved in \cite{graves2013generating}, where handwriting is generated using a mixture density network (MDN) \cite{bishop1994mixture}. These networks provide a way to model the most probable outcome as a function of multiple density terms. In order to predict from a history of input values, a Long short term memory (LSTM) \cite{hochreiter1997long} is included. This allows a way to store previous inputs and therefore predict outcomes when similarities are found. People based trajectory prediction has previously been achieved \cite{alahi2016social} using a similar approach. Making use of LSTM and MDN we therefore predict trajectories using unsupervised learning, requiring only positions of people as an input.




\todo{expand section with some work of vision based object detection, convolutional networks and other related work that was of interest for the selected approach}